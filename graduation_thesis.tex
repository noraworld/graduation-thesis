\documentclass[12pt, a4paper]{jsarticle}

\usepackage[dvipdfmx]{graphicx}
\usepackage{here}
\usepackage{comment}
\usepackage{otf}
\usepackage{framed}

% \title{タイトル}
% \author{}
% \date{}

\begin{document}
% \maketitle

% ----------------------------------------------------------------------------------------------------

\section*{第1章 はじめに}
\newpage














































\begin{comment}
表の挿入はここから
\begin{table}[H]
	\caption{表のタイトルを入力}
	\label{tb: example1}
	\begin{center}
		\begin{tabular}{ | c | c | c | } \hline
			タイトル & タイトル & タイトル \\ \hline\hline
			内容1-1 & 内容1-2 & 内容1-3 \\ \hline
			内容2-1 & 内容2-2 & 内容2-3 \\ \hline
		\end{tabular}
	\end{center}
\end{table}
表の挿入はここまで
表\ref{tb: example1}のように表番号を参照することができます.
\end{comment}

\begin{comment}
図の挿入はここから
\begin{figure}[H]
	\begin{center}
		\includegraphics[width=(好きな数値)mm]{画像ファイル名を入力}
	\end{center}
	\caption{図のタイトルを入力}
	\label{fig: example1}
\end{figure}
図の挿入はここまで
図\ref{fig: example1}のように図番号を参照することができます.
\end{comment}





\end{document}